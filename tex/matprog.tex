%%%%%%%%%%%%%%%%%%%%%%%%%%%%%%%%%%%%%%%%%%%%%%%%%%%%%%%%%%%%%%%%%%%
% Author: Shimon Panfil                                           %
% Copyright (c) Shimon Panfil Industrial Physics and Simulations  %
% http://industrialphys.com                                       %
%                                                                 %
% THE SOFTWARE IS PROVIDED "AS IS", WITHOUT WARRANTY OF ANY KIND, %
% EXPRESS OR IMPLIED, INCLUDING BUT NOT LIMITED TO THE WARRANTIES % 
% OF MERCHANTABILITY, FITNESS FOR A PARTICULAR PURPOSE AND        %
% NONINFRINGEMENT. IN NO EVENT SHALL THE AUTHORS OR COPYRIGHT     %
% HOLDERS BE LIABLE FOR ANY CLAIM, DAMAGES OR OTHER LIABILITY.    %
%%%%%%%%%%%%%%%%%%%%%%%%%%%%%%%%%%%%%%%%%%%%%%%%%%%%%%%%%%%%%%%%%%%
\documentclass{beamer}
\usepackage[utf8x]{inputenc}
\mode<presentation>
{
  \usetheme{default}
}
\title{Mathematical Programming with Fortran}
\author[Shimon Panfil]{Shimon Panfil,Ph. D.}
\institute{Industrial Physics and Simulations \\
    http://industrialphys.com/}
\date{\today}
\begin{document}
\frame{\titlepage}
\frame{
    \frametitle{Contents}
    \tableofcontents
}

\section{Mathematical Programming}
\subsection{Problem}
\frame {
\frametitle{Problem}
    Mathematical Programming (MP) means the solution
    of large and difficult scientific and engineering problems with computers. 
    "Large and difficult" are problems that cannot be solved without programming, by using Matlab, Excel and the like, e.g.
    solving 3D PDE.
    
    MP is the oldest field of programming, actually Fortran (which stands for {\em for}mula 
    {\em tran}slation) - the first ever created programming language was designed
    with such kind of problems in mind. It was developed over a  period from 1954 to 1957 
    by an IBM team, lead by John Backus. Both MP and Fortran are still  under active  development
    mainly in large computing centers belonging to both universities and industry.
}
\frame {
\frametitle{Problem}
    Intensive growth of business, multimedia, internet and other non-mathematical
    applications  has overshadowed MP, so  most of modern computer professionals
    and users know nothing about it. More importantly, most of the people teaching
    programming and writing books on the subject do not know about MP either. So, their
    recommendations on programming style and methodology are in many cases counterproductive
    when applied to Mathematical Programming.

    However, the most important fact is that modern programming languages like C, C++, Java,
    \ldots were designed primarily for non-mathematical applications. So we have a dilemma:
    use Fortran which is convenient for MP, but does not exploit modern computer architecture
    or use a modern programming language, one less suitable for MP.
}
\subsection {Solution}
\frame {
\frametitle{Solution}
    There are a number of solutions for the problem described above:
    \begin{enumerate}
	\item Use supercomputer with High Performance Fortran. Not discussed here.
	\item Write mathematics in C. Needs proficiency both in
	mathematics and programming. Not discussed here.   
	\item Use modern Fortran. 
    \end{enumerate}
    
}
\section {Modern Fortran}
\subsection{Fortran History}
\frame{
\frametitle{Fortran History}
    
    \begin{description}
	\item [Fortran 66] First standard, reasonably machine independent programming
	    language with efficient implementation.
	\item [Fortran 77] Most of existing code complies to this standard.
	\item [Fortran 90] Adds: pointers, modules, recursion, dynamic storage allocation.
	\item [Fortran 95] Marks a number of F77 features as obsolete.
	\item [Fortran 2003] Adds object-oriented programming support and C compatibility.
	\item [Fortran 2008] Adds Co-arrays and BIT data type.
    \end{description}     
}
\subsection {F03 Features}
\frame {
  \frametitle{F03 Features}
   \begin{itemize}
	\item Derived type enhancements
	\item Object-oriented programming support: type extension and inheritance, 
	    polymorphism, dynamic type allocation, and type-bound procedures.
	\item Data manipulation enhancements:  VOLATILE attribute,  pointer enhancements, extended 
	    initialization expressions, and enhanced intrinsic procedures.
	\item Input/output enhancements: asynchronous transfer, stream access ... 
	\item Procedure pointers.
	\item Support for IEEE floating-point arithmetic and floating point exception handling 
	\item Interoperability with the C programming language.
	\item Support for international usage
	\item Enhanced integration with the host operating system
	\end{itemize}
}
\subsection{State of the Art}
\frame{
\frametitle{State of the Art}
    Language standard is not enough for practical programming. One also needs implementation:
    compiler, runtime system, libraries \ldots . Before 2005, the only good available Fortran
    for PC was the g77 from gnu compiler collection (gcc), which implements F77 standard with some
    extensions. Development of the replacement was started in 2000 (g95 project). In 2003
    gfortran forked from g95 and became a part of gcc from version 4.0. It includes support for the
    Fortran 95 language and is compatible with most language extensions supported by g77, allowing
    it to serve as a drop-in replacement in many cases. Parts of Fortran 2003 and Fortran 2008
    have also been implemented. A number of commercial compilers also exists. Here, F77 will
    refer to g77 and F03 to gfortran. Fortran will stand for both F77 and F03.

}
\section {Optimization}
\subsection{Optimization Myths}
\frame {
\frametitle{Optimization Myths}
    There are a lot of myths about programming languages' efficiency and optimization.
    Anybody can do simple experiment: write a program which multiplies 2 matrices in C,
    Fortran, and Matlab. Matrices should be large enough, otherwise the effect will be too small.
    You will see that C and Fortran code have approximately the same speed, Matlab script
    is much slower, but Matlab's built-in function is much faster. So, in this case Fortran is not
    essentially more effective than C and is less effective than properly used Matlab. However,
    if you use intrinsic matrix multiplication in Fortran you'll get speed similar to Matlab's 
    built-in function. For a problem like 3D PDE solution Fortran (and C) will outperform Matlab.
    The answer is that there are at least 3 levels of optimization.
}
\begin{frame} [fragile] 
\begin{verbatim}
program mm3
    integer :: n,cac,i,j,k
    real,allocatable, dimension(:,:) :: a,b,c1,c2
    CHARACTER(len=32) :: arg
    real::start_time,stop_time
    real, parameter:: eps=1.0e-15
    cac=command_argument_count()
    if(cac/=1) then
        print *,"need 1 argument: dimension"
        stop
    end if    
    CALL get_command_argument(1, arg)
    read (arg,*) n
    print *,"dimension=",n
\end{verbatim}
\end{frame}
\begin{frame} [fragile] 
\begin{verbatim}
 allocate(a(n,n),b(n,n),c1(n,n),c2(n,n))
    call random_seed()
    call random_number(a)
    call random_number(b)  
    call cpu_time(stop_time)
    print *,"matrix preparation",stop_time-start_time,"s"
    call cpu_time(start_time)
    c1=matmul(a,b)
    call cpu_time(stop_time)
    print *,"matmul",stop_time-start_time,"s"

\end{verbatim}
\end{frame}
\begin{frame} [fragile] 
\begin{verbatim}
  call cpu_time(start_time)
    c1=0.0
    do k=1,n
        do i=1,n
            do j=1,n
                c1(i,j)=c1(i,j)+a(i,k)*b(k,j)
            end do
        end do
    end do
    call cpu_time(stop_time)
    print *,"loop kij",stop_time-start_time,"s"
    if(any(abs(c1-c2)>eps)) then
        print *,"different values, stop"
        stop
    end if
end program mm3
\end{verbatim}
\end{frame}

\begin{frame} [fragile] 
\begin{verbatim}
 dimension=         500
 matrix preparation  9.99899999999999921E-003 s
 matmul  0.12999100000000000      s
 loop ijk  0.29664799999999997      s
 loop ikj   3.0064700000000002      s
 loop jik  0.30331399999999986      s
 loop jki  0.15332299999999988      s
 loop kji  0.18998800000000005      s
 loop kij   3.0031369999999997      s
\end{verbatim}
\end{frame}

\begin{frame} [fragile] 
\begin{verbatim}
dimension=        1000
 matrix preparation  4.33300000000000005E-002 s
 matmul   1.7032219999999998      s
 loop ijk   11.222601000000001      s
 loop ikj   31.081305999999998      s
 loop jik   11.659240000000004      s
 loop jki   1.9398739999999961      s
 loop kji   2.7431539999999970      s
 loop kij   31.221298000000004      s
\end{verbatim}
\end{frame}
\begin{frame} [fragile] 
\begin{verbatim}
dimension=        1500
 matrix preparation  9.66589999999999949E-002 s
 matmul   5.7096280000000004      s
 loop ijk   45.273714000000005      s
 loop ikj   109.26620900000000      s
 loop jik   45.953670999999986      s
 loop jki   6.5329069999999945      s
 loop kji   9.6693699999999865      s
 loop kij   108.81957099999997      s
\end{verbatim}
\end{frame}
\begin{frame} [fragile] 
\begin{verbatim}
 dimension=        2000
 matrix preparation  0.16998800000000000      s
 matmul   13.462455000000000      s
 loop ijk   114.15922400000001      s
 loop ikj   280.43838099999999      s
 loop jik   116.78571899999997      s
 loop jki   15.608983000000080      s
 loop kji   22.701852999999915      s
 loop kij   280.77169299999991      s
\end{verbatim}
\end{frame}
\begin{frame}
\frametitle{Dynamic vs. Static}
    Here  the results of two similar programs are compared for $n=1000$.
    The only difference is that one use allocatable arrays and another
    static ones.

\begin{tabular}{|c|c|c|} \hline
algorithm & dynamic & static\\ \hline
mm  & 1.70  & 1.70 \\ \hline 
ijk & 11.22 & 11.54 \\ \hline
ikj & 31.16 & 31.29 \\ \hline
jik & 11.66 & 11.78 \\ \hline
jki & 1.94  & 1.56  \\ \hline
kji & 2.79  & 2.24  \\ \hline
kij & 32.25 & 31.60 \\ \hline
\end{tabular}

\end{frame}
\subsection{Compiler Optimization}
\frame {
\frametitle{Compiler Optimization}
    Large and difficult topic. All modern compilers comprise of
    3 main parts
    \begin{enumerate}
    \item Language dependent front-end.
    \item Intermediate representation "middle-end".
    \item Platform dependent back-end.
    \end{enumerate}
    Optimization can occur during any phase of compilation, however, the bulk of 
    optimizations are performed in the "middle-end"--- after the syntax and semantic analysis of the 
    front-end and before the code generation of the back-end. So optimization is mostly
    language and architecture independent. Fortran front-end passes more information to
    middle-end but the compiler is smart enough to infer the necessary information in relatively
    simple cases.
}
\subsection{Memory Optimization}
\frame {
\frametitle{Memory Optimization}
    Compiler performs well when restricted to processor optimization. Modern
    computers have memory hierarchy: fast and expensive (hence small) memory near processor
    core (L1 cache), slower and larger L2 cache etc. up to main memory which is the slowest.
    Optimization of cache usage is generally beyond compiler abilities. Actually, too 
     aggressive compiler optimization may spoil cache usage optimization. BTW, it is just this
    optimization that is responsible for results of our experiment: both Matlab built-in and Fortran
    intrinsic functions may be optimized for cache usage.
}
\subsection{Algorithm Optimization}
\frame {
\frametitle{Algorithm Optimization}
    This is the bread and butter of mathematical programming: the choice of 
    algorithm or combination of algorithms which works best for problem at hand.
    Simple example: you want to solve linear system $A x=b$. The theoretical answer
    is $x=A^{-1} b$ and you can use highly optimized algorithms for matrix inversion.
    However, a much better solution is LU decomposition of matrix A with subsequent 
    back-substitution.

    The program may comprise of more than one algorithm. Even if all algorithms
    are the most optimal, their combination may not be if, for example, large amount
    of data must be rearranged between subsequent parts.
}
\section {Fortran in Mathematical Programming}
\subsection {Why Fortran?}
\frame {
\frametitle{Why Fortran?}
    So why Fortran is good for MP (say better than C)?
    \begin{description}
    \item [compiler optimization?] No, practically no difference.
    \item [cache optimization?] No, it's language neutral.
    \item [algorithms?] Yes, modern Fortran (F03) allows writing mathematical
	algorithms in the most natural, and so less error prone, way!
    \end {description}
}
\subsection {Expressive Power}
\frame {
\frametitle{Expressive Power}
    Fortran now has all the features that one expects in a modern programming language:
    \begin{itemize}
     \item Arrays  can be used in expressions, as arguments to intrinsic
	functions, and in assignment statements; there is also a concise array-section notation.
     \item Dynamic storage is fully supported, with both automatic and allocatable arrays; 
	pointers can be used to handle more complex objects such as linked-lists and trees.
     \item Derived data types (data structures) are fully supported.
     \item The MODULE is a  program unit which fully supports encapsulation of data and 
	procedures and greatly facilitates code re-use.
     \item Procedures can be recursive or generic, they can have optional or keyword arguments,
	arrays can be passed complete with size/shape information and functions can return
	arrays or data structures.
     \end{itemize}
}
\frame {
\frametitle{Expressive Power}
    \begin{itemize}
     \item New operators can be defined, or existing operators can be overloaded for use on objects
	of derived type.
     \item Modules and data structures may declare their contents PRIVATE to enhance encapsulation
	and avoid name-space pollution.
     \end{itemize}
     So Fortran is as expressive as C++, but arrays and complex numbers are first-class objects,
     not derived ones - they are part of the language. Highly optimized and well tested intrinsic 
     procedures exist to work with these objects. In a sense modern Fortran combines the  power 
     of Matlab to express mathematical algorithm with the efficiency of C. 
}
\section {example}
\begin {frame} [fragile] 
\frametitle {Example}
    This Fortran 2003 program caculates  2d heat transfer over rectangular plate
    using relaxation method.
\begin{verbatim}

program heat_transfer
    implicit none
    integer ::m,n,c
    CHARACTER(len=32) :: arg
    real ::coeff
    real, allocatable, dimension(:,:), target :: plate
    real, allocatable, dimension(:,:) :: temp
    real, pointer, dimension(:,:) :: north, east, south, west, inside
    real, parameter :: tolerance =1.0e-4    
    real :: diff
    integer ::i,j,niter

\end{verbatim}
\end{frame}

\begin {frame} [fragile]
\frametitle {Example}
\begin{verbatim}
    c=command_argument_count()
    if(c/=2) then
        print *,"need 2 arguments"
        stop
    end if    
    CALL get_command_argument(1, arg)
    read (arg,'(I10)') m
    CALL get_command_argument(2, arg)
    read (arg,'(I10)') n
    allocate(plate(m,n))
    allocate(temp(m-2,n-2))
\end{verbatim}
\end{frame}
\begin {frame} [fragile]
\frametitle {Example}
\begin{verbatim}
    !initial conditions
    plate=0
    !boundary conditions
    plate(1:m,1)=1.0
    coeff=1.0/n
    plate(1,1:n)=[(coeff*j,j=n,1,-1)]
\end{verbatim}
\end{frame}

\begin{frame} [fragile]
\frametitle {Example}
    Pointers to different parts of the same array.
    Fortran pointers are not merely addresses, they
    have full information on index triplet start:stop:stride.
\begin{verbatim}
    inside=>plate(2:m-1,2:n-1)
    north=>plate(1:m-2,2:n-1)
    south=>plate(3:m,2:n-1)
    east=>plate(2:m-1,1:n-2)
    west=>plate(2:m-1,3:n)
\end{verbatim}
\end{frame}

\begin{frame} [fragile]
\frametitle {Example}
    Main loop (relaxation), no indices at all!
\begin{verbatim}
    niter=0
    do
        temp=0.25*(north+east+south+west)
        diff=maxval(abs(temp-inside))
        niter=niter+1
        inside=temp
        if(diff<tolerance) then
            exit
        endif
    end do
    print *,plate(m/2,:)
end program heat_transfer
\end{verbatim}
\end{frame}

\end{document}

